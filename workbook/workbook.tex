\documentclass[11pt, a4paper, draft, titlepage]{article}

\begin{document}

\title{ Git In Here\\ \large An Unofficial Introduction \\To Git Source
Control Management } \author{Sam Dixon} \date{2018} \maketitle

\begin{abstract} The purpose of this documentation is to provide a simple and
concise introduction to the use and understanding of the Git source control
management system.

While the goal of this material is to promote and educate the use of Git, it
is aimed at version control beginners and Git newcomers, and as such some of
the more complex or convoluted functions of Git will be omitted.

By following the exercises and advice provided in this guide, a user should be
able to gain an understanding of how Git operates, as well as the required
commands to version control their own projects and collaborate with other Git
users.

This guide and its partner presentation are released under the MIT license,
allowing you to alter, amend and redistribute them as you see fit - provided
attribution to the original author is maintained.  \end{abstract}

\tableofcontents 
\pagebreak


\section{What Is Version Control?}

\subsection{Background} Version Control (also known as Reversion Control or
Source Control) is a system that tracks and manages changes to a project.  A
project can be any collection of files on a PC, from a program, website or
even a plain text document, Version Control can be utilised to organise and
control any alterations made to a project.  A number of Version Control
Systems (VCSs) are available, each with their own benefits, features, and
unique quirks; Git, CVS, Subversion, and Mercurial are just a few of the more
commonly used VCSs.

\subsection{Key Terms} VCSs may be diverse in style and functionality, but
there does exist some shared features across multiple systems. While the
terminology may differ, the following list contains some of the most common
features present in the majority of VCSs:

\begin{itemize} 
\item Repository \\ The folder or directory that contains the
project.  A Repository can be local (on your PC), or remote (on a server).
\item Clone \\ The act of creating a local copy of a remote Repository.  
\item Commit \\ The act of finalising a change to Repository.  
\item Pull/Push \\ Allows users to share changes made to a local Repository to
a remote, or vice versa.  
\item Branch \\ A Repository can have multiple branches or forks,
allowing development of features to progress independently of each other.
\item Merge \\ Allows multiple Branches, and their components to be combined.
\end{itemize}

Note: the previous list is just a simple summary of terms, each feature and
its use will be described in richer detail later in this guide.


\section{Why use Version Control?} To a novice user, the previous section's
descriptions and terms may sound convoluted and complicated, and the described
functionality might be achieved with a USB memory-sticks or cloud storage
solutions - so why should you bother to learn this new system?

You might find version control useful if you have ever found yourself doing
any of the following; 
\begin{itemize}
\item Create multiple copies of a folder "In case I need the old version".
\item Copied work on to a USB drive, Dropbox, or Google Drive to move it from
one PC to another.
\item Copied code into an email or instant message to send it to a friend or
co-worker.
\item Lost work due to your laptop running out of charge or a hard drive
failing.
\item Overwritten someone's work when collaborating on the same project.
\end{itemize}

All the experiences listed, and more, can be avoided via the correct practice
of version control, potentially saving you time and effort when working on
projects of any size.

Whether you need a system that allows you to mange multiple versions or drafts
of a project, lets you work on a project across multiple devices, or provides a
secure off-site back up, version control can provide all these features and more
in single system.


\section{Why Git?} As previously stated, there is a wide range of VCS systems
to choose from, so why does this guide focus on Git?

Git Source Control Manager (SCM) is a free, open-source, distributed VCS
primarily designed to be used for the management of source-code and
programming projects.  The first version of Git was written in 2005 by Linus
Torvalds (creator of the Linux kernel), and has since been adopted as the most
commonly used source control system among developers.  Git is not only popular
amongst open-source developers, many industry leading corporations and
organisations also utilise Git as their preferred version control system.
Examples of companies that use Git include; Google, Intel, Microsoft, Amazon,
and Apple, to name a few.


The reasons this guide teaches Git, rather than another VCS are;
\begin{itemize} 
\item Freedom \\ Git is free to use and open-source, meaning
the source code is publicly available to view.  
\item Multiplatform \\ While
it was primarily designed to be used with Linux distributions, Git works just
as well on Windows and macOS, with all the same commands and functionality.
\item Distributed \\ Git allows you to maintain your repositories both on your
local machine or on a remote server maintained by a large number of providers
(You can even host your own personal Git server).  
\item Popularity \\ Git has
been recognized as one of, if not the most popular VCS among developers and
organisations, making Git an attractive skill to learn.  
\end{itemize}

\subsection{Git Tools} There exists a number of tools, plug-ins and add-ons
that claim to optimise or streamline the use of Git.  Tools like GitKraken,
Atlassian Sorucetree and Github Desktop all provide 

\section{Setting Up Git}

\section{Creating A Repository}

\section{Staging Files}

\section{Committing Changes}

\section{Pushing}


\end{document}

