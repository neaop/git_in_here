\documentclass[11pt, a4paper, draft, titlepage]{article}
\usepackage{hyperref}

\begin{document}

\title{ Git In Here\\
\large An Unofficial Introduction 
\\To Git Source Control Management
}

\author{Sam Dixon\\
\texttt{\href{mailto:s.dixon@napier.ac.uk}{s.dixon@napier.ac.uk}}
}

\date{2018} 
\maketitle

\begin{abstract} The purpose of this documentation is to provide a simple and
concise introduction to the use and understanding of the Git source control
management system.

While the goal of this material is to promote and educate the use of Git, it
is aimed at version control beginners and Git newcomers, and as such some of
the more complex or convoluted functions of Git will be omitted.

By following the exercises and advice provided in this guide, a user should be
able to gain an understanding of how Git operates, as well as the required
commands to version control their own projects and collaborate with other Git
users.

This guide and its partner presentation are released under the MIT license,
allowing you to alter, amend and redistribute them as you see fit - provided
attribution to the original author is maintained. 
\end{abstract}


\tableofcontents 
\pagebreak


\section{What Is Version Control?}
Perhaps your have heard terms like version control or Git slung about, either
praising there usefulness or despairing at there complexity. 
Regardless about the context, you may have asked yourself: 
``Why should I bother''.
Maybe you save all your work on Dropbox or Google Drive, or you might have a
handy-dandy USB key-chain that you never leave home without that you use to
ensure all your projects are up-to-date and accessible.
Regardless of your previous knowledge of Version Control, or the current
manner you use to store and access your work, this document aims to highlight
the benefits of using Version Control, and hopefully motivate you to learn and
use the tolls described henceforth.

A Version Control System (also known as Reversion Control, Source Control, or
VCS) is a tool that tracks and manages changes made to a project.  
A project can be any collection of folders and files on a PC, from program
source-code, website content and style sheets, or even a simple text report,
Version Control can be utilised to organise and control any alterations made
to the files within a project. 
A number of Version Control Systems (VCSs) are available, each with their own
benefits, features, and unique quirks; Git, CVS, Subversion, and Mercurial are
just a few of the more popular VCSs.


\subsection{Key Features}
It is difficult to justify the effort of learning a new tool if you do not
know what it is capable of.
While different VCSs may offer variations, the following is a list of key
features typical of VCSs:
\begin{itemize}
\item History\\
VCSs track changes that have been made to a file and who authored them.
This timeline of changes can let you see when and how files have been edited,
helping you locate when a bug was introduced to project or which team members
worked on a particular feature.
\item Reversion\\
VCSs don't just let you track a project's history, they let you revisit it. 
A file can be reverted to its previous remove a change, or the entire project
can be rolled back to a previous version.
\item Branching\\
VCSs allow for multiple versions or branches of a project to exist
simultaneously. 
These branches allow a project to be developed asynchronously, one branch can
be dedicated to patching bugs or adding featers to a system, while another
branch consists only of stable code that can be released to the public.
\item Tagging\\
VCSs provide a system that allows users to mark certain points in a project's
history as important.
This tagging system can be used in a number of ways; from notify project
releases (version 1.0, version 2.0, ...) to marking changes that introduced
bugs, or simply to place a bookmark in you project's history.
\item Collaboration\\
Arguably the most important feature of a VCS is that it allows multiple
parties work on a single project at the same time.
The manner in which a VCS implements collaboration can vary drastically, but
systems will provide users a way in which they can edit files without the
fear of erasing one another's work or overwriting important data.
\end{itemize}



\section{Why Use Version Control?} 
To a novice user, the previous section's descriptions and terms may sound
convoluted and complicated, and the described functionality might be achieved
with USB pen drives, portable hard drives, or cloud storage solutions - so why
should you bother to learn this new system?  
You might find version control useful if you have ever found yourself doing
any of the following;
\begin{itemize}
\item Created multiple copies of a project, ``In case I need the old version''.
\item Copied work on to a USB drive, Dropbox, or Google Drive to move it from
one PC to another.
\item Copied code into an email or instant message to send it to a friend or
co-worker.
\item Lost work due to your laptop running out of charge or a hard drive
failing.
\item Overwritten someone's work when collaborating on the same project.
\end{itemize}

All the experiences listed, and more, can be avoided via the correct practice
of version control - potentially saving you time and effort when working on
projects of any size.



\section{Why Git?} 
As previously stated, there is a wide range of VCS systems to choose from, so
why does this guide focus on Git?


Git Source Control Manager (SCM) is a free, open-source, distributed VCS
primarily designed to be used for the management of source-code and
programming projects.  The first version of Git was written in 2005 by Linus
Torvalds (creator of the Linux kernel), and has since been adopted as the most
commonly used source control system among developers [cite].

Git is not only popular amongst open-source developers, many industry leading
corporations and organisations also utilise Git as their preferred version
control system.  Examples of companies that use Git include; Google, Intel,
Microsoft, Amazon, and Apple, to name a few.

The following is a list of some of the reaons you might choose to learn Git
over another VCS:
\begin{itemize} 
\item Cost \\
Git is completely free to download and use, regardless of the size your
project or team.  
While some Git providers may charge you to host a project on
their hardware, there are plenty of free alternatives, and there is notheing
stopping you running your own personal Git server.  
\item Multiplatform \\ 
While it was primarily designed to be used with Linux distributions, Git works
just as well on Windows and macOS, with all the same commands and functionality.
\item Distributed \\ 
Git allows you to maintain your repositories both on your local machine or on
a remote server maintained by a large number of providers. 
This means you do not need an active internet connection or constant
"check-ins" while you are working on a project.
\item Speed \\
Git was purposely designed to be light-weight to run and to provide fast
execution of commands, even on large projects.
\item Popularity \\ 
Git has been recognized as one of, if not the most popular VCS among
developers and organisations, meaning there is a wealth of documentation
available for the learning and use of Git. 
It's popularity can also make Git an attractive skill in terms of
employability.
\end{itemize}


\subsection{Git Tools} There exists a number of tools, plug-ins and add-ons
that can aid users by optimising or streamlining the use of Git.  
Tools like GitKraken,  SourceTree, GitHub Desktop, and Tower all
provide visual Graphical User Interfaces (GUIs) - providing users with an
alternative to the typical terminal interface.

\section{Setting Up Git}

\section{Creating A Repository}

\section{Staging Files}

\section{Committing Changes}

\section{Pushing}


\end{document}

