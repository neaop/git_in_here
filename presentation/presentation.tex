\documentclass{beamer}
\usecolortheme{solarized}
\usepackage{hyperref}
\setbeamertemplate{section in toc}[sections numbered]


\title{Git In Here}
\subtitle{An Unoffical Introduction To Git Version Control Management}
\author{Sam Dixon\\
{\tt \href{mailto:s.dixon@napier.ac.uk}{s.dixon@napier.ac.uk}}}
\date{2018}
\titlegraphic{\includegraphics[height=3.5cm]{../res/git_icon_orange.png}} 


\begin{document}


\frame{\titlepage}


\begin{frame}{Contents}
    \tableofcontents
\end{frame}


\section{What is Version Control?}
\begin{frame}[allowframebreaks]{\secname}
    Version Control Systems are tools that let you track changes    
    
    A number of different VCSs are available:
    \begin{itemize}
    \item Subversion
    \item Mercurial
    \item CVS
    \item Fossil
    \item Git
    \end{itemize}
    Each VCS has its own quirks in style and implementation 

\framebreak

    Common features of version control include:
    \begin{itemize}
    \item Change History
    \item Change Reversion  
    \item Branching
    \item Tagging
    \item Collaboration
    \end{itemize}
\end{frame}


\begin{frame}{Why you should use Version control}
    VCS can help with common problems like:

    \begin{itemize}
    \item Creating multiple copies files ``in case you need the old one''
    \item Put work onto a USB drive to move it between computers
    \item Copied code into email / IM to send it to a friend
    \item Lost work due to loss of power / faulty hard drive
    \item Overwritten someone else's work when collaborating
    \end{itemize}
\end{frame}


\section{What is Git?}
\begin{frame}[allowframebreaks]{\secname}
    Git was developed in 2005 by Linus Torvalds.
    
    It has become one of the most popular VCSs in use today.

    Some companies that use Git include:
    \begin{itemize}
    \item Google
    \item Apple
    \item Microsoft
    \item Facebook
    \item Amazon
    \item Adobe
    \item Mozilla
    \item NASA
    \end{itemize}

\framebreak

    Git has a number of key features that make it an attractive VCS
    \begin{itemize}
    \item Distributed
    \item Decentralised
    \item Free (like beer and like speech)
    \item Multiplatform
    \item Speed
    \item Popularity
    \end{itemize}

\end{frame}


\section{Setting up Git}
\begin{frame}[allowframebreaks]{\secname}
    Git can be installed in a number of ways:
    \begin{itemize}
    \item Download directly from Git ({\tt \url{https://git-scm.com/downloads}}) 
    \item Install a Git visualiser (GitHub Desktop, Sourcetree, Kraken)
    \item Install via a package manager 
    \end{itemize}

\framebreak

    Git requires a name and email to create commits \\~\\

    {\tt \$ git congif --global user.name "Sam"}

    {\tt \$ git config --global user.email sam@mail.com} \\~\\

    These settings are not used as login credentials / tokens

\end{frame}

\section{Common Terms}
\begin{frame}{\secname}
    Git has a number of key terms that you should keep in mind
    \begin{itemize}
    \item Repository 
    \item Commit
    \item Push / Pull
    \item Tag
    \item Branch
    \item Merge
    \end{itemize}
    Some of terms may be used in other VCSs, but their functions may not be
    the same

\end{frame}

\section{Git Workflow}
\begin{frame}{\secname}
    The standard process for using Git is as follows: \\~\\
    \begin{enumerate}
        \item {\bf Initialise} a Git repo
        \item Do some work (edit files)
        \item {\bf Stage} the changes you want to keep
        \item Finalise the changes in a {\bf Commit}
        \item Repeat from step 2
    \end{enumerate}
\end{frame}

\section{Basic Commands}
\begin{frame}{Getting a Repo}
    
    Git can create a repo in as a new folder:\\~\\

    {\tt \$ git init project} \\~\\

    or in an existing directory: \\~\\

    {\tt \$ git init} \\~\\

    It doesn't matter if there are files already in the directory

\end{frame}

\begin{frame}{Staging Files}
    Git will not track files by default - you have to specify: \\~\\

    {\tt \$ git add file\_name} \\~\\

    You can add multiple files in a single command: \\~\\

    {\tt \$ git add file\_1 file\_2} \\~\\

    Or you can use wildcards: \\~\\

    {\tt \$ git add *.txt}
\end{frame}

\begin{frame}{Checking File State}
    You can check the state of a repo with the following: \\~\\

    {\tt \$ git status} \\~\\
    
    Files can be in multiple states:
    \begin{itemize}
    \item Untracked
    \item Modified
    \item Staged
    \item Deleted
    \end{itemize}
\end{frame}

\begin{frame}{Committing Changes}
    When you are ready to finalise your changes; commit: \\~\\

    {\tt \$ git commit} \\~\\

    You can also declare a commit message in the same command: \\~\\

    {\tt \$ git commit -m "Your message here"}
\end{frame}









\section{Tagging}
\begin{frame}[allowframebreaks]{\secname}
    Tags are simply a pointer to a specific commit

    Used to show commits of interest, like version releases or bugs \\~\\

    Create a lightweight tag with the following command:

    {\tt \$ git tag tag\_name} \\~\\

    Annotated tags can be created as well: 

    {\tt \$ git tag -a tag\_name -m "tag message"} \\~\\

\framebreak
    You can see what tags there are in a repo:

    {\tt \$ git tag} \\~\\

    Or you can inspect a specific tag:

    {\tt \$ git show tag\_name}


\end{frame}

\section{Branching}
\begin{frame}{\secname}
    Branches a key feature of Git - They allow different version of the same
    project to exists simultaneously

    Branches can be used to experiment or write patches

    Branches are inexpensive to create / delete
\end{frame}

\end{document}

