\documentclass[11pt, a4paper, titlepage]{article}
\usepackage{handout}


\begin{document}

\title{ 
    \LARGE Git In Here\\
    \Large An Unofficial Introduction\\
    To Git Source Control Management\\
    \bigbreak
    \includegraphics[scale=0.4]{../res/git_icon}}


\author{Sam Dixon\\
\texttt{\href{mailto:s.dixon@napier.ac.uk}{s.dixon@napier.ac.uk}}
}

\date{2018} 
\maketitle



\begin{abstract} The purpose of this documentation is to provide a simple and
concise introduction to the use and understanding of the Git source control
management system.

While the goal of this material is to promote and educate the use of Git, it
is aimed at version control beginners and Git newcomers, and as such some of
the more complex or convoluted functions of Git will be omitted.

By following the exercises and advice provided in this guide, a user should be
able to gain an understanding of how Git operates, as well as the required
commands to version control their own projects and collaborate with other Git
users.

This guide and its partner presentation are released under the MIT license,
allowing you to alter, amend and redistribute them as you see fit - provided
attribution to the original author is maintained. 
\end{abstract}



\tableofcontents 
\pagebreak



\section{What Is Version Control?}
Perhaps your have heard terms like version control or Git slung about, either
praising there usefulness or despairing at there complexity. 
Regardless about the context, you may have asked yourself: 
``Why should I bother''.
Maybe you save all your work on Dropbox or Google Drive, or you might have a
handy-dandy USB key-chain that you never leave home without that you use to
ensure all your projects are up-to-date and accessible.
Regardless of your previous knowledge of Version Control, or the current
manner you use to store and access your work, this document aims to highlight
the benefits of using Version Control, and hopefully motivate you to learn and
use the tolls described henceforth.

A Version Control System (also known as Reversion Control, Source Control, or
VCS) is a tool that tracks and manages changes made to a project.  
A project can be any collection of folders and files on a PC, from program
source-code, website content and style sheets, or even a simple text report,
Version Control can be utilised to organise and control any alterations made
to the files within a project. 
A number of Version Control Systems (VCSs) are available, each with their own
benefits, features, and unique quirks; Git, CVS, Subversion, and Mercurial are
just a few of the more popular VCSs.


\subsection{Features}
It is difficult to justify the effort of learning a new tool if you do not
know what it is capable of.
While different VCSs may offer variations, the following is a list of key
features typical of VCSs:

\begin{itemize}
\item History\\
VCSs track changes that have been made to a file and who authored them.
This timeline of changes can let you see when and how files have been edited,
helping you locate when a bug was introduced to project or which team members
worked on a particular feature.
\item Reversion\\
VCSs don't just let you track a project's history, they let you revisit it. 
A file can be reverted to its previous remove a change, or the entire project
can be rolled back to a previous version.
\item Branching\\
VCSs allow for multiple versions or branches of a project to exist
simultaneously. 
These branches allow a project to be developed asynchronously, one branch can
be dedicated to patching bugs or adding features to a system, while another
branch consists only of stable code that can be released to the public.
\item Tagging\\
VCSs provide a system that allows users to mark certain points in a project's
history as important.
This tagging system can be used in a number of ways; from notify project
releases (version 1.0, version 2.0, ...) to marking changes that introduced
bugs, or simply to place a bookmark in you project's history.
\item Collaboration\\
Arguably the most important feature of a VCS is that it allows multiple
parties work on a single project at the same time.
The manner in which a VCS implements collaboration can vary drastically, but
systems will provide users a way in which they can edit files without the
fear of erasing one another's work or overwriting important data.
\end{itemize}



\section{Why Use Version Control?} 
To a novice user, the previous section's descriptions and terms may sound
convoluted and complicated, and the described functionality might be achieved
with USB pen drives, portable hard drives, or cloud storage solutions - so why
should you bother to learn this new system?  
You might find version control useful if you have ever found yourself doing
any of the following;

\begin{itemize}
\item Created multiple copies of a project, ``In case I need the old version''.
\item Copied work on to a USB drive, Dropbox, or Google Drive to move it from
one PC to another.
\item Copied code into an email or instant message to send it to a friend or
co-worker.
\item Lost work due to your laptop running out of charge or a hard drive
failing.
\item Overwritten someone's work when collaborating on the same project.
\end{itemize}

All the experiences listed, and more, can be avoided via the correct practice
of version control - potentially saving you time and effort when working on
projects of any size.



\section{Why Git?} 
As previously stated, there is a wide range of VCS systems to choose from, so
why does this guide focus on Git?

Git Source Control Manager (SCM) is a free, open-source, distributed VCS
primarily designed to be used for the management of source-code and
programming projects.  The first version of Git was written in 2005 by Linus
Torvalds (creator of the Linux kernel), and has since been adopted as the most
commonly used source control system among developers [cite].

Git is not only popular amongst open-source developers, many industry leading
corporations and organisations also utilise Git as their preferred version
control system.  Examples of companies that use Git include; Google, Intel,
Microsoft, Amazon, and Apple, to name a few.


\subsection{Advantages}
The following is a list of some of the reasons you might choose to learn Git
over another VCS:
\begin{itemize} 
\item Cost \\
Git is completely free to download and use, regardless of the size your
project or team.  
While some Git providers may charge you to host a project on
their hardware, there are plenty of free alternatives, and there is nothing
stopping you running your own personal Git server.  
\item Multiplatform \\ 
While it was primarily designed to be used with Linux distributions, Git works
just as well on Windows and macOS, with all the same commands and functionality.
\item Distributed \\ 
Git allows you to maintain your repositories both on your local machine or on
a remote server maintained by a large number of providers. 
This means you do not need an active internet connection or constant
"check-ins" while you are working on a project.
\item Speed \\
Git was purposely designed to be light-weight to run and to provide fast
execution of commands, even on large projects.
\item Popularity \\ 
Git has been recognized as one of, if not the most popular VCS among
developers and organisations, meaning there is a wealth of documentation
available for the learning and use of Git. 
It's popularity can also make Git an attractive skill in terms of
employability.
\item Tools \\
While git is designed to be used from a terminal, a number of GUIs and
visualisers have been developed to provide users with an alternative interface
when using Git.
\end{itemize}

\subsection{Key Terms} 
Up to this point, we have been using generic terms like `project' and
`changes', but Git uses some specific terms for these concepts.
The following list contains definitions of key terms that will be used from
this point on.

\begin{itemize} 
\item Repository \\ 
The folder or directory that contains the project. 
A Repository can be local (on your PC), or remote (on a server).
\item Clone \\ 
The act of creating a local copy of a remote repository.
\item Staging \\
The act of selecting which changes will be will be included in the next
commit.
\item Commit \\ 
The act of finalising a change to Repository.  
\item Pull/Push \\ 
Allows users to share changes made to a local repository to a remote, or vice
versa.  
\item Branch \\ 
A Repository can have multiple branches or forks, allowing development of
features to progress independently of each other.
\item Merge \\ 
The act of combining the content of multiple branches onto a single branch.
\end{itemize}

Note: the previous list is just a simple summary of terms, each feature and
its use will be described in richer detail later in this guide.



\section{Setting Up Git}
This section will describe the steps required to get Git operating on your own
machine. 


\subsection{Installation}
Depending on your operating system, the process of installing git can vary,
below is a quick and easy install guide for Windows, macOS and Linux
distributions.

\subsubsection{Windows}
The simplest way to install Git on windows is to download the official Git-Bash
client directly from: \url{https://git-scm.com/download/win}.

Many Git GUIs and visualisers will also come packaged with Git-Bash when they
installed.

\subsubsection{macOS}
A macOS version of the git installer is available at:
\url{https://git-scm.com/download/mac}.
Much like Windows, many of the GUIs designed for Git will also install the
required Git-Bash.

\subsubsection{Linux}
Depending on your distribution, Git may already come pre-installed on your
machine, otherwise it can be installed simply using your package manager.
The Git-SCM website provides detailed installation instruction, for each
flavour of Linux: \url{https://git-scm.com/download/linux}


\subsection{Git Providers}
Due to Git's distributed nature you are perfectly able to install Git and
experience all of it features and functions. 
If however you want to be able to
share your projects with your peers, or access your code from anywhere in the
world you will need to host your repositories on a remote server.

A number of organisations will allow you to host your projects on their
servers, and while the underlying Git infrastructure may be identical, the
providers themselves may offer you additional services or features.

It is worth doing a little research to find which provider will work best for
you, but some of the more common services include
\href{https://github.com}{GitHub}, \href{https://gitlab.com}{GitLab}, and
\href{https://bitbucket.org}{Bitbucket}.


\subsection{Git Configuration}
Once you have Git installed on your machine you will need to perform a few
configuration steps in order to use Git at its full potential.

\subsubsection{Git Identity}
In order to create commits, Git requires two pieces of information, a user
name and email, without setting these variables you will be unable to publish
any changes you make with Git.
In order to set you name and email the following two commands can be run:

\begin{lstlisting}[label=lst_config_user,
caption=Updating the default Git identity]
 git config --global user.name "John Smith"
 git config --global user.email john.smith@email.com
\end{lstlisting}

\noindent
Lets go over the structure of the commands, and what they did.

\begin {itemize}
\item
All git commands a prefaced by the word {\tt git}, this identifies that
following command is to be run by Git, rather than another application
installed on the computer.
\item
The word {\tt config} is used with any commands that relate to the
configuration or setup of git.
\item
The {\tt --global} flag means that the following configuration will be applied
by default to all git repositories on the machine.
If you wanted to use a different email or username for a repository, you can
simply run the previous commands while within the project directory, omitting
the {\tt --global} flag.
\item
Obviously, the {\tt user.name} and {\tt user.email} commands alter the current
Git identities name and email.
If you are working on publicly visible code and are uncomfortable with
revealing your name you can use a pseudonym, and many Git providers will
assign you with an anonymous email which you can use for publishing commits.
\end{itemize}


\subsubsection{Text Editor}
Executing certain commands in Git will cause the system to launch a text
editor, allowing you to enter information like a tag or commit message.
Git will select a text editor for you automatically, which on a windows is VIM.
While VIM is a powerful tool in its own right, it is not the most intuitive,
and as such it is recommended that you change the editor to one you are more
familiar with.
To change the default editor to Nano, the following command can be used:

\begin{lstlisting}[label=lst_config_nano,
caption=Updating the defualt Git editor to Nano]
 git config --global core.editor "nano"
\end{lstlisting}

You are not just limited to terminal based editors either, you can have git
launch any text editor installed on your machine.
The following listing give an example of how to alter Git to run  Notepad++ as
the default editor;

\begin{lstlisting}[label=lst_config_note,
caption=Updating the defualt Git editor to Notepad++]
 git config --global core.editor "'C:/Program Files/Notepad++/notepad++.exe'-multiInst -nosession" 
\end{lstlisting}

Note that if you have installed Notepad++ in a different location, or want to
use a different editor you will have to amend the command appropriately.



\section{Using Git}
Now that Git has been successfully installed, and an identity has been created
to author commits, it is now possible to start using Git proper.

Before getting into the nitty-gritty, we are first going to have a brief
description of how Git works and the standard process of version controlling
using Git. 

Please note that the following descriptions are quite high-level and are
intended to help those who haven't used Git before better conceptualise the
process. A number of resources exist that go into much finer detail of the
technical implementation of Git.

\subsection{How Git Works}
At its core, Git simply allows a user to take a snapshot of the current state
of the files contained within a repository.
These snapshots make Git somewhat unique in terms of VCS's, which typically
only track the changes the changes made in each file.
The fact Git tracks snapshots of your project, rather than just the changes
allows it to operate in an entirely decentralised manner, and allows it to
perform some complex operations that other VCS's are unable to.

\subsubsection{File States}
There are three main states that a file can be in while in a Git repository:
\begin{itemize}
\item Committed - When a file is committed it means that its current state has
been saved by Git.
\item Modified - This means that the file has been changed since the last time
it was committed.
\item Staged - A stage file is one that has been changed you want to include
it in your next commit.
\end{itemize}

From a users point of view, a simplified Git workflow consists of making
changes to files, staging the files when you are happy with there current
state, and finalising the changes by committing them to Git's local database.

With the prvious description in mind, lets describe the actual commands needed
to carry out the workflow.

\section{Creating A Repository}
Now that we have Git installed and configured we can start to use Git proper.
In your terminal, navigate to a directory that you want to become a Git
repository - the parent directory of a project is ideal.
Once you are within the correct directory, execute the following command:

\begin{lstlisting}[label=lst_init,
caption=Initialisin a new Git repository]
 git init 
\end{lstlisting}

The {\tt init} command will create a hidden {\tt .git} directory within your
current directory.
This {\tt .git} directory contains all the files that  Git requires to track
files, changes, commits, branches, and more.

You can call the {\tt init} command in an empty directory, or in a folder that
already contains a project.



\section{Repository Status}
Now that we have initialised a Git repository, what can we do with it?
The status of a Git repository can be checked by executing the following
command while within a Git repository:

\begin{lstlisting}[label=lst_status,
caption=Initialisin a new Git repository]
 git status
\end{lstlisting}



\section{Staging Files}



\section{Committing Changes}



\section{Pushing}



\end{document}

